% Options for packages loaded elsewhere
\PassOptionsToPackage{unicode}{hyperref}
\PassOptionsToPackage{hyphens}{url}
%
\documentclass[
]{article}
\usepackage{amsmath,amssymb}
\usepackage{lmodern}
\usepackage{iftex}
\ifPDFTeX
  \usepackage[T1]{fontenc}
  \usepackage[utf8]{inputenc}
  \usepackage{textcomp} % provide euro and other symbols
\else % if luatex or xetex
  \usepackage{unicode-math}
  \defaultfontfeatures{Scale=MatchLowercase}
  \defaultfontfeatures[\rmfamily]{Ligatures=TeX,Scale=1}
\fi
% Use upquote if available, for straight quotes in verbatim environments
\IfFileExists{upquote.sty}{\usepackage{upquote}}{}
\IfFileExists{microtype.sty}{% use microtype if available
  \usepackage[]{microtype}
  \UseMicrotypeSet[protrusion]{basicmath} % disable protrusion for tt fonts
}{}
\makeatletter
\@ifundefined{KOMAClassName}{% if non-KOMA class
  \IfFileExists{parskip.sty}{%
    \usepackage{parskip}
  }{% else
    \setlength{\parindent}{0pt}
    \setlength{\parskip}{6pt plus 2pt minus 1pt}}
}{% if KOMA class
  \KOMAoptions{parskip=half}}
\makeatother
\usepackage{xcolor}
\usepackage[margin=1in]{geometry}
\usepackage{color}
\usepackage{fancyvrb}
\newcommand{\VerbBar}{|}
\newcommand{\VERB}{\Verb[commandchars=\\\{\}]}
\DefineVerbatimEnvironment{Highlighting}{Verbatim}{commandchars=\\\{\}}
% Add ',fontsize=\small' for more characters per line
\usepackage{framed}
\definecolor{shadecolor}{RGB}{248,248,248}
\newenvironment{Shaded}{\begin{snugshade}}{\end{snugshade}}
\newcommand{\AlertTok}[1]{\textcolor[rgb]{0.94,0.16,0.16}{#1}}
\newcommand{\AnnotationTok}[1]{\textcolor[rgb]{0.56,0.35,0.01}{\textbf{\textit{#1}}}}
\newcommand{\AttributeTok}[1]{\textcolor[rgb]{0.77,0.63,0.00}{#1}}
\newcommand{\BaseNTok}[1]{\textcolor[rgb]{0.00,0.00,0.81}{#1}}
\newcommand{\BuiltInTok}[1]{#1}
\newcommand{\CharTok}[1]{\textcolor[rgb]{0.31,0.60,0.02}{#1}}
\newcommand{\CommentTok}[1]{\textcolor[rgb]{0.56,0.35,0.01}{\textit{#1}}}
\newcommand{\CommentVarTok}[1]{\textcolor[rgb]{0.56,0.35,0.01}{\textbf{\textit{#1}}}}
\newcommand{\ConstantTok}[1]{\textcolor[rgb]{0.00,0.00,0.00}{#1}}
\newcommand{\ControlFlowTok}[1]{\textcolor[rgb]{0.13,0.29,0.53}{\textbf{#1}}}
\newcommand{\DataTypeTok}[1]{\textcolor[rgb]{0.13,0.29,0.53}{#1}}
\newcommand{\DecValTok}[1]{\textcolor[rgb]{0.00,0.00,0.81}{#1}}
\newcommand{\DocumentationTok}[1]{\textcolor[rgb]{0.56,0.35,0.01}{\textbf{\textit{#1}}}}
\newcommand{\ErrorTok}[1]{\textcolor[rgb]{0.64,0.00,0.00}{\textbf{#1}}}
\newcommand{\ExtensionTok}[1]{#1}
\newcommand{\FloatTok}[1]{\textcolor[rgb]{0.00,0.00,0.81}{#1}}
\newcommand{\FunctionTok}[1]{\textcolor[rgb]{0.00,0.00,0.00}{#1}}
\newcommand{\ImportTok}[1]{#1}
\newcommand{\InformationTok}[1]{\textcolor[rgb]{0.56,0.35,0.01}{\textbf{\textit{#1}}}}
\newcommand{\KeywordTok}[1]{\textcolor[rgb]{0.13,0.29,0.53}{\textbf{#1}}}
\newcommand{\NormalTok}[1]{#1}
\newcommand{\OperatorTok}[1]{\textcolor[rgb]{0.81,0.36,0.00}{\textbf{#1}}}
\newcommand{\OtherTok}[1]{\textcolor[rgb]{0.56,0.35,0.01}{#1}}
\newcommand{\PreprocessorTok}[1]{\textcolor[rgb]{0.56,0.35,0.01}{\textit{#1}}}
\newcommand{\RegionMarkerTok}[1]{#1}
\newcommand{\SpecialCharTok}[1]{\textcolor[rgb]{0.00,0.00,0.00}{#1}}
\newcommand{\SpecialStringTok}[1]{\textcolor[rgb]{0.31,0.60,0.02}{#1}}
\newcommand{\StringTok}[1]{\textcolor[rgb]{0.31,0.60,0.02}{#1}}
\newcommand{\VariableTok}[1]{\textcolor[rgb]{0.00,0.00,0.00}{#1}}
\newcommand{\VerbatimStringTok}[1]{\textcolor[rgb]{0.31,0.60,0.02}{#1}}
\newcommand{\WarningTok}[1]{\textcolor[rgb]{0.56,0.35,0.01}{\textbf{\textit{#1}}}}
\usepackage{graphicx}
\makeatletter
\def\maxwidth{\ifdim\Gin@nat@width>\linewidth\linewidth\else\Gin@nat@width\fi}
\def\maxheight{\ifdim\Gin@nat@height>\textheight\textheight\else\Gin@nat@height\fi}
\makeatother
% Scale images if necessary, so that they will not overflow the page
% margins by default, and it is still possible to overwrite the defaults
% using explicit options in \includegraphics[width, height, ...]{}
\setkeys{Gin}{width=\maxwidth,height=\maxheight,keepaspectratio}
% Set default figure placement to htbp
\makeatletter
\def\fps@figure{htbp}
\makeatother
\setlength{\emergencystretch}{3em} % prevent overfull lines
\providecommand{\tightlist}{%
  \setlength{\itemsep}{0pt}\setlength{\parskip}{0pt}}
\setcounter{secnumdepth}{-\maxdimen} % remove section numbering
\ifLuaTeX
  \usepackage{selnolig}  % disable illegal ligatures
\fi
\IfFileExists{bookmark.sty}{\usepackage{bookmark}}{\usepackage{hyperref}}
\IfFileExists{xurl.sty}{\usepackage{xurl}}{} % add URL line breaks if available
\urlstyle{same} % disable monospaced font for URLs
\hypersetup{
  pdftitle={Introduction to R, Data types and Data Structure in R},
  pdfauthor={:``Siddharth Chaudhary''},
  hidelinks,
  pdfcreator={LaTeX via pandoc}}

\title{Introduction to R, Data types and Data Structure in R}
\usepackage{etoolbox}
\makeatletter
\providecommand{\subtitle}[1]{% add subtitle to \maketitle
  \apptocmd{\@title}{\par {\large #1 \par}}{}{}
}
\makeatother
\subtitle{Data in R, Fall 2022}
\author{:``Siddharth Chaudhary''}
\date{}

\begin{document}
\maketitle

\hypertarget{section}{%
\section{--------------}\label{section}}

\hypertarget{getting-help}{%
\subsection{Getting help}\label{getting-help}}

\hypertarget{section-1}{%
\section{--------------}\label{section-1}}

\begin{Shaded}
\begin{Highlighting}[]
\FunctionTok{help}\NormalTok{(sqrt)}
\end{Highlighting}
\end{Shaded}

\begin{verbatim}
## starting httpd help server ... done
\end{verbatim}

\begin{Shaded}
\begin{Highlighting}[]
\NormalTok{?sqrt}
\FunctionTok{help.start}\NormalTok{()}
\end{Highlighting}
\end{Shaded}

\begin{verbatim}
## If nothing happens, you should open
## 'http://127.0.0.1:20875/doc/html/index.html' yourself
\end{verbatim}

\hypertarget{section-2}{%
\section{--------------}\label{section-2}}

\hypertarget{basic-operations}{%
\subsection{basic operations}\label{basic-operations}}

\hypertarget{section-3}{%
\section{--------------}\label{section-3}}

\begin{Shaded}
\begin{Highlighting}[]
\DecValTok{2}\SpecialCharTok{+}\DecValTok{2}
\end{Highlighting}
\end{Shaded}

\begin{verbatim}
## [1] 4
\end{verbatim}

\begin{Shaded}
\begin{Highlighting}[]
\DecValTok{2{-}3}\SpecialCharTok{*}\DecValTok{4}
\end{Highlighting}
\end{Shaded}

\begin{verbatim}
## [1] -10
\end{verbatim}

\begin{Shaded}
\begin{Highlighting}[]
\NormalTok{(}\DecValTok{2{-}3}\NormalTok{)}\SpecialCharTok{*}\DecValTok{4}
\end{Highlighting}
\end{Shaded}

\begin{verbatim}
## [1] -4
\end{verbatim}

\begin{Shaded}
\begin{Highlighting}[]
\DecValTok{5}\SpecialCharTok{/}\DecValTok{9}\SpecialCharTok{*}\DecValTok{4}
\end{Highlighting}
\end{Shaded}

\begin{verbatim}
## [1] 2.222222
\end{verbatim}

\begin{Shaded}
\begin{Highlighting}[]
\DecValTok{3}\SpecialCharTok{\^{}}\DecValTok{2}   \CommentTok{\# 3 to the power of 2}
\end{Highlighting}
\end{Shaded}

\begin{verbatim}
## [1] 9
\end{verbatim}

\begin{Shaded}
\begin{Highlighting}[]
\DecValTok{5} \SpecialCharTok{\textless{}} \DecValTok{4}
\end{Highlighting}
\end{Shaded}

\begin{verbatim}
## [1] FALSE
\end{verbatim}

\begin{Shaded}
\begin{Highlighting}[]
\DecValTok{3} \SpecialCharTok{==} \DecValTok{3}   \CommentTok{\# this is a test, are these values equal?}
\end{Highlighting}
\end{Shaded}

\begin{verbatim}
## [1] TRUE
\end{verbatim}

\begin{Shaded}
\begin{Highlighting}[]
\DecValTok{6}\SpecialCharTok{/}\DecValTok{3} \SpecialCharTok{!=} \DecValTok{2}  \CommentTok{\# this is a test, are these values NOT equal}
\end{Highlighting}
\end{Shaded}

\begin{verbatim}
## [1] FALSE
\end{verbatim}

\begin{Shaded}
\begin{Highlighting}[]
\DecValTok{8}\SpecialCharTok{/}\DecValTok{2} \SpecialCharTok{\textless{}=} \DecValTok{4}  \CommentTok{\# this is a test, is the left side value less than or equal to the right side?}
\end{Highlighting}
\end{Shaded}

\begin{verbatim}
## [1] TRUE
\end{verbatim}

\hypertarget{section-4}{%
\section{--------------}\label{section-4}}

\hypertarget{basic-functions}{%
\subsection{Basic functions}\label{basic-functions}}

\hypertarget{section-5}{%
\section{--------------}\label{section-5}}

\begin{Shaded}
\begin{Highlighting}[]
\FunctionTok{sqrt}\NormalTok{(}\DecValTok{4}\NormalTok{)   }\CommentTok{\#square root of four}
\end{Highlighting}
\end{Shaded}

\begin{verbatim}
## [1] 2
\end{verbatim}

\begin{Shaded}
\begin{Highlighting}[]
\FunctionTok{sqrt}\NormalTok{(}\DecValTok{9}\NormalTok{)   }\CommentTok{\#notice that comments after the pound sign are ignored}
\end{Highlighting}
\end{Shaded}

\begin{verbatim}
## [1] 3
\end{verbatim}

\begin{Shaded}
\begin{Highlighting}[]
\CommentTok{\# notice that case matters in R}
\FunctionTok{sqrt}\NormalTok{(}\DecValTok{9}\NormalTok{)}
\end{Highlighting}
\end{Shaded}

\begin{verbatim}
## [1] 3
\end{verbatim}

\begin{Shaded}
\begin{Highlighting}[]
\FunctionTok{print}\NormalTok{(}\StringTok{"Help! I\textquotesingle{}ve been abducted by aliens!!!"}\NormalTok{)}
\end{Highlighting}
\end{Shaded}

\begin{verbatim}
## [1] "Help! I've been abducted by aliens!!!"
\end{verbatim}

\begin{Shaded}
\begin{Highlighting}[]
\FunctionTok{cat}\NormalTok{(}\StringTok{"Help!}\SpecialCharTok{\textbackslash{}n\textbackslash{}n}\StringTok{"}\NormalTok{, }\StringTok{"I\textquotesingle{}ve been abducted by smart aliens!!!"}\NormalTok{, }\AttributeTok{sep=}\StringTok{""}\NormalTok{)}
\end{Highlighting}
\end{Shaded}

\begin{verbatim}
## Help!
## 
## I've been abducted by smart aliens!!!
\end{verbatim}

\begin{Shaded}
\begin{Highlighting}[]
\NormalTok{?cat}
\end{Highlighting}
\end{Shaded}

Notice that the output dumps onto the console. Whatever is dumped to the
console is not retained.

We can change that by assigning the results to a object.

\begin{Shaded}
\begin{Highlighting}[]
\NormalTok{my\_results }\OtherTok{\textless{}{-}} \FunctionTok{sqrt}\NormalTok{(}\DecValTok{9}\NormalTok{)}
\NormalTok{my\_results}
\end{Highlighting}
\end{Shaded}

\begin{verbatim}
## [1] 3
\end{verbatim}

\begin{Shaded}
\begin{Highlighting}[]
\CommentTok{\#My\_results   \# case matters}

\NormalTok{my\_results }\OtherTok{=} \FunctionTok{sqrt}\NormalTok{(}\DecValTok{16}\NormalTok{)}
\NormalTok{my\_results}
\end{Highlighting}
\end{Shaded}

\begin{verbatim}
## [1] 4
\end{verbatim}

\begin{Shaded}
\begin{Highlighting}[]
\CommentTok{\# Better practice is to use assignment operator}
\end{Highlighting}
\end{Shaded}

\hypertarget{section-6}{%
\section{--------------}\label{section-6}}

\hypertarget{command-separation}{%
\subsection{command separation}\label{command-separation}}

\hypertarget{section-7}{%
\section{--------------}\label{section-7}}

\begin{Shaded}
\begin{Highlighting}[]
\FunctionTok{print}\NormalTok{(}\StringTok{"blah"}\NormalTok{)}
\end{Highlighting}
\end{Shaded}

\begin{verbatim}
## [1] "blah"
\end{verbatim}

\begin{Shaded}
\begin{Highlighting}[]
\FunctionTok{print}\NormalTok{(}\StringTok{"ughh"}\NormalTok{)}
\end{Highlighting}
\end{Shaded}

\begin{verbatim}
## [1] "ughh"
\end{verbatim}

\begin{Shaded}
\begin{Highlighting}[]
\FunctionTok{print}\NormalTok{(}\StringTok{"really?"}\NormalTok{)}
\end{Highlighting}
\end{Shaded}

\begin{verbatim}
## [1] "really?"
\end{verbatim}

\begin{Shaded}
\begin{Highlighting}[]
\FunctionTok{print}\NormalTok{(}\StringTok{"blah"}\NormalTok{); }\FunctionTok{print}\NormalTok{(}\StringTok{"ughh"}\NormalTok{); }\FunctionTok{print}\NormalTok{(}\StringTok{"really?"}\NormalTok{)  }\CommentTok{\# semi colons for separation}
\end{Highlighting}
\end{Shaded}

\begin{verbatim}
## [1] "blah"
\end{verbatim}

\begin{verbatim}
## [1] "ughh"
\end{verbatim}

\begin{verbatim}
## [1] "really?"
\end{verbatim}

\begin{Shaded}
\begin{Highlighting}[]
\CommentTok{\# we can also string a command across multiple lines, though this is not good practice}

\CommentTok{\# notice that R adds a + sign to the new line to indicate continuation of the command}
\FunctionTok{print}\NormalTok{(}\StringTok{"blah"}
\NormalTok{)}
\end{Highlighting}
\end{Shaded}

\begin{verbatim}
## [1] "blah"
\end{verbatim}

In this lecture, you will learn about the most \emph{basic}
\textbf{data} structure in R.

Render code as `code here'. \# -------------- \# Part 1 - Basic data
types \# --------------

There are four common types of data in R: double (numeric), integer,
logical and character. (We won't work with raw or complex values.)

\hypertarget{section-8}{%
\section{--------------}\label{section-8}}

\hypertarget{numeric}{%
\subsection{Numeric}\label{numeric}}

\hypertarget{section-9}{%
\section{--------------}\label{section-9}}

\textbf{double} or \textbf{numeric} vector

\begin{Shaded}
\begin{Highlighting}[]
\NormalTok{vect\_dbl }\OtherTok{\textless{}{-}} \FunctionTok{c}\NormalTok{(}\FloatTok{1.1}\NormalTok{,}\SpecialCharTok{{-}}\FloatTok{2.5}\NormalTok{,}\DecValTok{4}\NormalTok{,}\DecValTok{6}\NormalTok{,}\FloatTok{2.2}\NormalTok{,}\DecValTok{12}\NormalTok{) }\CommentTok{\# create vector data object}
\NormalTok{vect\_dbl}
\end{Highlighting}
\end{Shaded}

\begin{verbatim}
## [1]  1.1 -2.5  4.0  6.0  2.2 12.0
\end{verbatim}

\begin{Shaded}
\begin{Highlighting}[]
\FunctionTok{typeof}\NormalTok{(vect\_dbl)}
\end{Highlighting}
\end{Shaded}

\begin{verbatim}
## [1] "double"
\end{verbatim}

We can also examine a data object using the \textbf{str} structure
function.

\begin{Shaded}
\begin{Highlighting}[]
\FunctionTok{str}\NormalTok{(vect\_dbl)}
\end{Highlighting}
\end{Shaded}

\begin{verbatim}
##  num [1:6] 1.1 -2.5 4 6 2.2 12
\end{verbatim}

Notice that the \textbf{str} function tells us this is numeric data!

\hypertarget{section-10}{%
\section{--------------}\label{section-10}}

\hypertarget{integer}{%
\subsection{integer}\label{integer}}

\hypertarget{section-11}{%
\section{--------------}\label{section-11}}

\textbf{integer} vector

\begin{Shaded}
\begin{Highlighting}[]
\NormalTok{vect\_int }\OtherTok{\textless{}{-}} \SpecialCharTok{{-}}\DecValTok{1}\SpecialCharTok{:}\DecValTok{10} \CommentTok{\# create vector data object}
\NormalTok{vect\_int}
\end{Highlighting}
\end{Shaded}

\begin{verbatim}
##  [1] -1  0  1  2  3  4  5  6  7  8  9 10
\end{verbatim}

\begin{Shaded}
\begin{Highlighting}[]
\FunctionTok{typeof}\NormalTok{(vect\_int)}
\end{Highlighting}
\end{Shaded}

\begin{verbatim}
## [1] "integer"
\end{verbatim}

\begin{Shaded}
\begin{Highlighting}[]
\FunctionTok{str}\NormalTok{(vect\_int)}
\end{Highlighting}
\end{Shaded}

\begin{verbatim}
##  int [1:12] -1 0 1 2 3 4 5 6 7 8 ...
\end{verbatim}

R treats integers the same as numeric values for most functions.

\hypertarget{section-12}{%
\section{--------------}\label{section-12}}

\hypertarget{logical}{%
\subsection{logical}\label{logical}}

\hypertarget{section-13}{%
\section{--------------}\label{section-13}}

\textbf{logical} vector

\begin{Shaded}
\begin{Highlighting}[]
\NormalTok{vect\_log }\OtherTok{\textless{}{-}} \FunctionTok{c}\NormalTok{(T, }\ConstantTok{FALSE}\NormalTok{, }\ConstantTok{TRUE}\NormalTok{, F,T,T) }\CommentTok{\# create vector data object}
\NormalTok{vect\_log}
\end{Highlighting}
\end{Shaded}

\begin{verbatim}
## [1]  TRUE FALSE  TRUE FALSE  TRUE  TRUE
\end{verbatim}

\begin{Shaded}
\begin{Highlighting}[]
\FunctionTok{typeof}\NormalTok{(vect\_log)}
\end{Highlighting}
\end{Shaded}

\begin{verbatim}
## [1] "logical"
\end{verbatim}

\begin{Shaded}
\begin{Highlighting}[]
\FunctionTok{str}\NormalTok{(vect\_log)}
\end{Highlighting}
\end{Shaded}

\begin{verbatim}
##  logi [1:6] TRUE FALSE TRUE FALSE TRUE TRUE
\end{verbatim}

\begin{Shaded}
\begin{Highlighting}[]
\FunctionTok{cat}\NormalTok{(}\StringTok{"}\SpecialCharTok{\textbackslash{}n}\StringTok{sum of logical vector: "}\NormalTok{, }\FunctionTok{sum}\NormalTok{(vect\_log))}
\end{Highlighting}
\end{Shaded}

\begin{verbatim}
## 
## sum of logical vector:  4
\end{verbatim}

Note that R treats T and TRUE, F and FALSE as equivalent. When treated
as numbers, T=1, F=0.

\textbf{logical} and numeric vector

\begin{Shaded}
\begin{Highlighting}[]
\NormalTok{vect\_log }\OtherTok{\textless{}{-}} \FunctionTok{c}\NormalTok{(T, }\ConstantTok{FALSE}\NormalTok{, }\ConstantTok{TRUE}\NormalTok{, F,T,T, }\FloatTok{5.5}\NormalTok{) }\CommentTok{\# create vector data object}
\NormalTok{vect\_log}
\end{Highlighting}
\end{Shaded}

\begin{verbatim}
## [1] 1.0 0.0 1.0 0.0 1.0 1.0 5.5
\end{verbatim}

\begin{Shaded}
\begin{Highlighting}[]
\FunctionTok{typeof}\NormalTok{(vect\_log)}
\end{Highlighting}
\end{Shaded}

\begin{verbatim}
## [1] "double"
\end{verbatim}

\begin{Shaded}
\begin{Highlighting}[]
\FunctionTok{str}\NormalTok{(vect\_log)}
\end{Highlighting}
\end{Shaded}

\begin{verbatim}
##  num [1:7] 1 0 1 0 1 1 5.5
\end{verbatim}

\begin{Shaded}
\begin{Highlighting}[]
\FunctionTok{cat}\NormalTok{(}\StringTok{"}\SpecialCharTok{\textbackslash{}n}\StringTok{sum of logical vector: "}\NormalTok{, }\FunctionTok{sum}\NormalTok{(vect\_log))}
\end{Highlighting}
\end{Shaded}

\begin{verbatim}
## 
## sum of logical vector:  9.5
\end{verbatim}

\hypertarget{section-14}{%
\section{--------------}\label{section-14}}

\hypertarget{character}{%
\subsection{character}\label{character}}

\hypertarget{section-15}{%
\section{--------------}\label{section-15}}

\textbf{character} vector

\begin{Shaded}
\begin{Highlighting}[]
\NormalTok{vect\_chr }\OtherTok{\textless{}{-}} \FunctionTok{c}\NormalTok{(}\StringTok{"Dave"}\NormalTok{, }\StringTok{"Vlad"}\NormalTok{,}\StringTok{"Chuck"}\NormalTok{,}\StringTok{"Larry"}\NormalTok{,}\StringTok{"Bob"}\NormalTok{,}\StringTok{"Mary"}\NormalTok{,}\StringTok{"Doug"}\NormalTok{,}\StringTok{"Daryll"}\NormalTok{,}\StringTok{"Ron"}\NormalTok{,}\StringTok{"Sue"}\NormalTok{, }\StringTok{"Laura"}\NormalTok{, }\StringTok{"Rich"}\NormalTok{)}
\NormalTok{vect\_chr}
\end{Highlighting}
\end{Shaded}

\begin{verbatim}
##  [1] "Dave"   "Vlad"   "Chuck"  "Larry"  "Bob"    "Mary"   "Doug"   "Daryll"
##  [9] "Ron"    "Sue"    "Laura"  "Rich"
\end{verbatim}

\begin{Shaded}
\begin{Highlighting}[]
\FunctionTok{typeof}\NormalTok{(vect\_chr)}
\end{Highlighting}
\end{Shaded}

\begin{verbatim}
## [1] "character"
\end{verbatim}

\begin{Shaded}
\begin{Highlighting}[]
\FunctionTok{str}\NormalTok{(vect\_chr)}
\end{Highlighting}
\end{Shaded}

\begin{verbatim}
##  chr [1:12] "Dave" "Vlad" "Chuck" "Larry" "Bob" "Mary" "Doug" "Daryll" ...
\end{verbatim}

Note that each new line in the vector printout is labeled with the
vector position. Character values are represented with quotes.

The structure function will list specific character values as space
allows, then\ldots{}

\textbf{character and numeric} vector

\begin{Shaded}
\begin{Highlighting}[]
\NormalTok{vect\_chr }\OtherTok{\textless{}{-}} \FunctionTok{c}\NormalTok{(}\StringTok{"Dave"}\NormalTok{, }\StringTok{"Vlad"}\NormalTok{,}\StringTok{"Chuck"}\NormalTok{,}\StringTok{"Larry"}\NormalTok{,}\StringTok{"Bob"}\NormalTok{,}\StringTok{"Mary"}\NormalTok{,}\StringTok{"Doug"}\NormalTok{,}\StringTok{"Daryll"}\NormalTok{,}\StringTok{"Ron"}\NormalTok{,}\StringTok{"Sue"}\NormalTok{, }\StringTok{"Laura"}\NormalTok{, }\StringTok{"Rich"}\NormalTok{, }\FloatTok{5.5}\NormalTok{)}
\NormalTok{vect\_chr}
\end{Highlighting}
\end{Shaded}

\begin{verbatim}
##  [1] "Dave"   "Vlad"   "Chuck"  "Larry"  "Bob"    "Mary"   "Doug"   "Daryll"
##  [9] "Ron"    "Sue"    "Laura"  "Rich"   "5.5"
\end{verbatim}

\begin{Shaded}
\begin{Highlighting}[]
\FunctionTok{typeof}\NormalTok{(vect\_chr)}
\end{Highlighting}
\end{Shaded}

\begin{verbatim}
## [1] "character"
\end{verbatim}

\begin{Shaded}
\begin{Highlighting}[]
\FunctionTok{str}\NormalTok{(vect\_chr)}
\end{Highlighting}
\end{Shaded}

\begin{verbatim}
##  chr [1:13] "Dave" "Vlad" "Chuck" "Larry" "Bob" "Mary" "Doug" "Daryll" ...
\end{verbatim}

\hypertarget{section-16}{%
\section{--------------}\label{section-16}}

\hypertarget{atomic-vector}{%
\subsection{Atomic vector}\label{atomic-vector}}

\hypertarget{section-17}{%
\section{--------------}\label{section-17}}

An \textbf{atomic vector} or \textbf{vector} in R is a one-dimensional
data structure. A \textbf{vector} holds a string of values, all of the
same type, with the string having (practically) any length. You can
create a vector using the combine or concatenate \textbf{c} function.

\begin{Shaded}
\begin{Highlighting}[]
\NormalTok{vect }\OtherTok{\textless{}{-}} \FunctionTok{c}\NormalTok{(}\DecValTok{1}\NormalTok{,}\SpecialCharTok{{-}}\DecValTok{2}\NormalTok{,}\DecValTok{4}\NormalTok{,}\DecValTok{6}\NormalTok{,}\FloatTok{2.2}\NormalTok{,}\DecValTok{12}\NormalTok{) }\CommentTok{\# create vector data object}
\NormalTok{vect  }\CommentTok{\# print data object to the console}
\end{Highlighting}
\end{Shaded}

\begin{verbatim}
## [1]  1.0 -2.0  4.0  6.0  2.2 12.0
\end{verbatim}

Notice how the vector is printed to the console.

You can test whether an object is an atomic vector using two different
commands.

\begin{Shaded}
\begin{Highlighting}[]
\FunctionTok{is.atomic}\NormalTok{(vect)   }\CommentTok{\# more robust test}
\end{Highlighting}
\end{Shaded}

\begin{verbatim}
## [1] TRUE
\end{verbatim}

\begin{Shaded}
\begin{Highlighting}[]
\FunctionTok{is.vector}\NormalTok{(vect)   }\CommentTok{\# won\textquotesingle{}t always work, depending upon attributes}
\end{Highlighting}
\end{Shaded}

\begin{verbatim}
## [1] TRUE
\end{verbatim}

\hypertarget{section-18}{%
\section{--------------}\label{section-18}}

\hypertarget{vector-properties}{%
\subsection{Vector properties}\label{vector-properties}}

\hypertarget{section-19}{%
\section{--------------}\label{section-19}}

We can assign and extract vector properties.

We can obtain the vector length with the following command.

\begin{Shaded}
\begin{Highlighting}[]
\CommentTok{\# length function gives length of a vector (number of values stored)}
\FunctionTok{length}\NormalTok{(vect)}
\end{Highlighting}
\end{Shaded}

\begin{verbatim}
## [1] 6
\end{verbatim}

\begin{Shaded}
\begin{Highlighting}[]
\FunctionTok{length}\NormalTok{(}\FunctionTok{c}\NormalTok{(}\StringTok{"123456789"}\NormalTok{,}\StringTok{"0"}\NormalTok{))}
\end{Highlighting}
\end{Shaded}

\begin{verbatim}
## [1] 2
\end{verbatim}

There are six pieces of data, six positions in this vector.

We can name elements in a vector.

\begin{Shaded}
\begin{Highlighting}[]
\CommentTok{\# Create a vector with my height and weight in SI units (m, kg)}
\NormalTok{vect\_example }\OtherTok{\textless{}{-}} \FunctionTok{c}\NormalTok{(}\FloatTok{1.75}\NormalTok{, }\DecValTok{70}\NormalTok{)}
\NormalTok{vect\_example}
\end{Highlighting}
\end{Shaded}

\begin{verbatim}
## [1]  1.75 70.00
\end{verbatim}

\begin{Shaded}
\begin{Highlighting}[]
\FunctionTok{cat}\NormalTok{(}\StringTok{"}\SpecialCharTok{\textbackslash{}n}\StringTok{"}\NormalTok{)   }\CommentTok{\# this prints a line space}
\end{Highlighting}
\end{Shaded}

\begin{Shaded}
\begin{Highlighting}[]
\CommentTok{\#now assign names for meaning, notice these are character strings}
\FunctionTok{names}\NormalTok{(vect\_example) }\OtherTok{\textless{}{-}} \FunctionTok{c}\NormalTok{(}\StringTok{"height (m)"}\NormalTok{, }\StringTok{"weight (kg)"}\NormalTok{)}
\NormalTok{vect\_example}
\end{Highlighting}
\end{Shaded}

\begin{verbatim}
##  height (m) weight (kg) 
##        1.75       70.00
\end{verbatim}

\begin{Shaded}
\begin{Highlighting}[]
\FunctionTok{cat}\NormalTok{(}\StringTok{"}\SpecialCharTok{\textbackslash{}n}\StringTok{"}\NormalTok{)   }\CommentTok{\# this prints a line space}
\end{Highlighting}
\end{Shaded}

\begin{Shaded}
\begin{Highlighting}[]
\FunctionTok{names}\NormalTok{(vect\_example) }\OtherTok{\textless{}{-}} \FunctionTok{c}\NormalTok{(}\StringTok{"height (m)"}\NormalTok{, }\StringTok{"weight (kg)"}\NormalTok{)}
\NormalTok{vect\_example}
\end{Highlighting}
\end{Shaded}

\begin{verbatim}
##  height (m) weight (kg) 
##        1.75       70.00
\end{verbatim}

We can also assign vector names when creating the vector.

\begin{Shaded}
\begin{Highlighting}[]
\CommentTok{\# Create a vector with my height and weight in SI units (m, kg)}
\NormalTok{vect\_example }\OtherTok{\textless{}{-}} \FunctionTok{c}\NormalTok{(}\StringTok{"height (m)"}\OtherTok{=}\FloatTok{1.75}\NormalTok{, }\StringTok{"weight(kg)"} \OtherTok{=}\DecValTok{70}\NormalTok{)}
\NormalTok{vect\_example}
\end{Highlighting}
\end{Shaded}

\begin{verbatim}
## height (m) weight(kg) 
##       1.75      70.00
\end{verbatim}

We can inquire about vector type using \textbf{typeof()} function.

\begin{Shaded}
\begin{Highlighting}[]
\FunctionTok{typeof}\NormalTok{(vect\_example)}
\end{Highlighting}
\end{Shaded}

\begin{verbatim}
## [1] "double"
\end{verbatim}

\begin{Shaded}
\begin{Highlighting}[]
\FunctionTok{typeof}\NormalTok{(}\FunctionTok{c}\NormalTok{(}\StringTok{"123456789"}\NormalTok{,}\StringTok{"0"}\NormalTok{))}
\end{Highlighting}
\end{Shaded}

\begin{verbatim}
## [1] "character"
\end{verbatim}

``double'' means double precision numeric or commonly thought of as
``numeric''. We will discuss other types of data.

Finally we can inquire about data object attributes.

\begin{Shaded}
\begin{Highlighting}[]
\NormalTok{vect }\OtherTok{\textless{}{-}} \FunctionTok{c}\NormalTok{(}\DecValTok{1}\NormalTok{,}\SpecialCharTok{{-}}\DecValTok{2}\NormalTok{,}\DecValTok{4}\NormalTok{,}\DecValTok{6}\NormalTok{,}\FloatTok{2.2}\NormalTok{,}\DecValTok{12}\NormalTok{) }\CommentTok{\# create vector data object}
\FunctionTok{attributes}\NormalTok{(vect)}
\end{Highlighting}
\end{Shaded}

\begin{verbatim}
## NULL
\end{verbatim}

\begin{Shaded}
\begin{Highlighting}[]
\FunctionTok{cat}\NormalTok{(}\StringTok{"}\SpecialCharTok{\textbackslash{}n}\StringTok{"}\NormalTok{)}
\end{Highlighting}
\end{Shaded}

\begin{Shaded}
\begin{Highlighting}[]
\FunctionTok{attributes}\NormalTok{(vect\_example)}
\end{Highlighting}
\end{Shaded}

\begin{verbatim}
## $names
## [1] "height (m)" "weight(kg)"
\end{verbatim}

The first vector has no attributes. The second has names. In object
oriented programming, data object attributes can be considered ``meta
data''.

\hypertarget{section-20}{%
\section{--------------}\label{section-20}}

\hypertarget{creating-vectors}{%
\subsection{Creating Vectors}\label{creating-vectors}}

\hypertarget{section-21}{%
\section{--------------}\label{section-21}}

You can also create vectors with other commands.

\begin{Shaded}
\begin{Highlighting}[]
\NormalTok{vect\_rep }\OtherTok{\textless{}{-}} \FunctionTok{rep}\NormalTok{(}\FloatTok{1.2}\NormalTok{,}\AttributeTok{times=}\DecValTok{10}\NormalTok{) }\CommentTok{\# replicate value 10 times}
\NormalTok{vect\_rep}
\end{Highlighting}
\end{Shaded}

\begin{verbatim}
##  [1] 1.2 1.2 1.2 1.2 1.2 1.2 1.2 1.2 1.2 1.2
\end{verbatim}

\begin{Shaded}
\begin{Highlighting}[]
\NormalTok{vect\_seq }\OtherTok{\textless{}{-}} \FunctionTok{seq}\NormalTok{(}\AttributeTok{from=}\DecValTok{1}\NormalTok{, }\AttributeTok{to=}\DecValTok{13}\NormalTok{, }\AttributeTok{by=}\DecValTok{2}\NormalTok{) }\CommentTok{\# sequence from 1 to 12 by 2}
\NormalTok{vect\_seq}
\end{Highlighting}
\end{Shaded}

\begin{verbatim}
## [1]  1  3  5  7  9 11 13
\end{verbatim}

\begin{Shaded}
\begin{Highlighting}[]
\NormalTok{vect\_seq\_by1 }\OtherTok{\textless{}{-}} \DecValTok{1}\SpecialCharTok{:}\DecValTok{6} \CommentTok{\# sequence integers by 1}
\NormalTok{vect\_seq\_by1}
\end{Highlighting}
\end{Shaded}

\begin{verbatim}
## [1] 1 2 3 4 5 6
\end{verbatim}

\begin{Shaded}
\begin{Highlighting}[]
\NormalTok{vect\_seq\_by1\_neg }\OtherTok{\textless{}{-}} \SpecialCharTok{{-}}\DecValTok{1}\SpecialCharTok{:}\DecValTok{6} \CommentTok{\# sequence integers by 1}
\NormalTok{vect\_seq\_by1\_neg}
\end{Highlighting}
\end{Shaded}

\begin{verbatim}
## [1] -1  0  1  2  3  4  5  6
\end{verbatim}

\begin{Shaded}
\begin{Highlighting}[]
\NormalTok{vect\_seq\_by1\_pr }\OtherTok{\textless{}{-}} \SpecialCharTok{{-}}\NormalTok{(}\DecValTok{1}\SpecialCharTok{:}\DecValTok{6}\NormalTok{) }\CommentTok{\# sequence integers by 1}
\NormalTok{vect\_seq\_by1\_pr}
\end{Highlighting}
\end{Shaded}

\begin{verbatim}
## [1] -1 -2 -3 -4 -5 -6
\end{verbatim}

\hypertarget{section-22}{%
\section{--------------}\label{section-22}}

\hypertarget{missing-values-na}{%
\subsection{Missing values (NA)}\label{missing-values-na}}

\hypertarget{section-23}{%
\section{--------------}\label{section-23}}

All spaces in a vector or matrix must be filled with something. If there
is no data for a position, that position is represented by the missing
value designation in R, \textbf{NA}. Note that a missing value is NOT
the same as the character value ``NA''.

Replace Doug with NA and Sue with ``NA''.

\begin{Shaded}
\begin{Highlighting}[]
\NormalTok{vect\_na }\OtherTok{\textless{}{-}} \FunctionTok{c}\NormalTok{(}\StringTok{"Dave"}\NormalTok{, }\StringTok{"Vlad"}\NormalTok{,}\StringTok{"Chuck"}\NormalTok{,}\StringTok{"Larry"}\NormalTok{,}\StringTok{"Bob"}\NormalTok{,}\StringTok{"Mary"}\NormalTok{,}\ConstantTok{NA}\NormalTok{,}\StringTok{"Daryll"}\NormalTok{,}\StringTok{"Ron"}\NormalTok{,}\StringTok{"NA"}\NormalTok{, }\StringTok{"Laura"}\NormalTok{, }\StringTok{"Rich"}\NormalTok{)}
\NormalTok{vect\_na}
\end{Highlighting}
\end{Shaded}

\begin{verbatim}
##  [1] "Dave"   "Vlad"   "Chuck"  "Larry"  "Bob"    "Mary"   NA       "Daryll"
##  [9] "Ron"    "NA"     "Laura"  "Rich"
\end{verbatim}

\begin{Shaded}
\begin{Highlighting}[]
\FunctionTok{str}\NormalTok{(vect\_na)}
\end{Highlighting}
\end{Shaded}

\begin{verbatim}
##  chr [1:12] "Dave" "Vlad" "Chuck" "Larry" "Bob" "Mary" NA "Daryll" "Ron" ...
\end{verbatim}

\begin{Shaded}
\begin{Highlighting}[]
\NormalTok{vect\_rep\_na }\OtherTok{\textless{}{-}} \FunctionTok{rep}\NormalTok{(}\ConstantTok{NA}\NormalTok{,}\DecValTok{85}\NormalTok{)}
\NormalTok{vect\_rep\_na}
\end{Highlighting}
\end{Shaded}

\begin{verbatim}
##  [1] NA NA NA NA NA NA NA NA NA NA NA NA NA NA NA NA NA NA NA NA NA NA NA NA NA
## [26] NA NA NA NA NA NA NA NA NA NA NA NA NA NA NA NA NA NA NA NA NA NA NA NA NA
## [51] NA NA NA NA NA NA NA NA NA NA NA NA NA NA NA NA NA NA NA NA NA NA NA NA NA
## [76] NA NA NA NA NA NA NA NA NA NA
\end{verbatim}

\begin{Shaded}
\begin{Highlighting}[]
\FunctionTok{str}\NormalTok{(vect\_rep\_na)}
\end{Highlighting}
\end{Shaded}

\begin{verbatim}
##  logi [1:85] NA NA NA NA NA NA ...
\end{verbatim}

\hypertarget{section-24}{%
\section{--------------}\label{section-24}}

\hypertarget{coerce-data-types}{%
\section{coerce data types}\label{coerce-data-types}}

\hypertarget{section-25}{%
\section{--------------}\label{section-25}}

We can \emph{coerce} data into a particular type or from one type to
another.

\hypertarget{as.character}{%
\subsection{as.character}\label{as.character}}

\textbf{as.character} examples

\begin{Shaded}
\begin{Highlighting}[]
\NormalTok{vect }\OtherTok{\textless{}{-}} \FunctionTok{c}\NormalTok{(}\FloatTok{1.1}\NormalTok{,}\SpecialCharTok{{-}}\FloatTok{2.5}\NormalTok{,}\DecValTok{4}\NormalTok{,}\DecValTok{6}\NormalTok{,}\FloatTok{2.2}\NormalTok{,}\DecValTok{12}\NormalTok{) }\CommentTok{\# create vector data object}
\FunctionTok{cat}\NormalTok{(}\StringTok{"numeric to character }\SpecialCharTok{\textbackslash{}n}\StringTok{"}\NormalTok{)  }\CommentTok{\# cat prints nicely}
\end{Highlighting}
\end{Shaded}

\begin{verbatim}
## numeric to character
\end{verbatim}

\begin{Shaded}
\begin{Highlighting}[]
\NormalTok{vect}
\end{Highlighting}
\end{Shaded}

\begin{verbatim}
## [1]  1.1 -2.5  4.0  6.0  2.2 12.0
\end{verbatim}

\begin{Shaded}
\begin{Highlighting}[]
\FunctionTok{as.character}\NormalTok{(vect)}
\end{Highlighting}
\end{Shaded}

\begin{verbatim}
## [1] "1.1"  "-2.5" "4"    "6"    "2.2"  "12"
\end{verbatim}

\begin{Shaded}
\begin{Highlighting}[]
\CommentTok{\# we put a \textbackslash{}n in front for an extra return or line}
\NormalTok{vect }\OtherTok{\textless{}{-}} \SpecialCharTok{{-}}\DecValTok{1}\SpecialCharTok{:}\DecValTok{8} \CommentTok{\# create vector data object}
\FunctionTok{cat}\NormalTok{(}\StringTok{"}\SpecialCharTok{\textbackslash{}n}\StringTok{integer to character}\SpecialCharTok{\textbackslash{}n}\StringTok{"}\NormalTok{)}
\end{Highlighting}
\end{Shaded}

\begin{verbatim}
## 
## integer to character
\end{verbatim}

\begin{Shaded}
\begin{Highlighting}[]
\NormalTok{vect}
\end{Highlighting}
\end{Shaded}

\begin{verbatim}
##  [1] -1  0  1  2  3  4  5  6  7  8
\end{verbatim}

\begin{Shaded}
\begin{Highlighting}[]
\FunctionTok{as.character}\NormalTok{(vect)}
\end{Highlighting}
\end{Shaded}

\begin{verbatim}
##  [1] "-1" "0"  "1"  "2"  "3"  "4"  "5"  "6"  "7"  "8"
\end{verbatim}

\begin{Shaded}
\begin{Highlighting}[]
\FunctionTok{cat}\NormalTok{(}\StringTok{"}\SpecialCharTok{\textbackslash{}n}\StringTok{"}\NormalTok{)}
\end{Highlighting}
\end{Shaded}

\begin{Shaded}
\begin{Highlighting}[]
\NormalTok{vect }\OtherTok{\textless{}{-}} \FunctionTok{c}\NormalTok{(T, }\ConstantTok{FALSE}\NormalTok{, }\ConstantTok{TRUE}\NormalTok{, F,T,T) }\CommentTok{\# create vector data object}
\FunctionTok{cat}\NormalTok{(}\StringTok{"}\SpecialCharTok{\textbackslash{}n}\StringTok{logical to character}\SpecialCharTok{\textbackslash{}n}\StringTok{"}\NormalTok{)}
\end{Highlighting}
\end{Shaded}

\begin{verbatim}
## 
## logical to character
\end{verbatim}

\begin{Shaded}
\begin{Highlighting}[]
\NormalTok{vect}
\end{Highlighting}
\end{Shaded}

\begin{verbatim}
## [1]  TRUE FALSE  TRUE FALSE  TRUE  TRUE
\end{verbatim}

\begin{Shaded}
\begin{Highlighting}[]
\FunctionTok{as.character}\NormalTok{(vect)}
\end{Highlighting}
\end{Shaded}

\begin{verbatim}
## [1] "TRUE"  "FALSE" "TRUE"  "FALSE" "TRUE"  "TRUE"
\end{verbatim}

\hypertarget{as.numeric}{%
\subsection{as.numeric}\label{as.numeric}}

\textbf{as.numeric} examples

\begin{Shaded}
\begin{Highlighting}[]
\FunctionTok{cat}\NormalTok{(}\StringTok{"integer to numeric}\SpecialCharTok{\textbackslash{}n}\StringTok{"}\NormalTok{)}
\end{Highlighting}
\end{Shaded}

\begin{verbatim}
## integer to numeric
\end{verbatim}

\begin{Shaded}
\begin{Highlighting}[]
\NormalTok{vect}
\end{Highlighting}
\end{Shaded}

\begin{verbatim}
## [1]  TRUE FALSE  TRUE FALSE  TRUE  TRUE
\end{verbatim}

\begin{Shaded}
\begin{Highlighting}[]
\FunctionTok{as.numeric}\NormalTok{(vect)}
\end{Highlighting}
\end{Shaded}

\begin{verbatim}
## [1] 1 0 1 0 1 1
\end{verbatim}

\begin{Shaded}
\begin{Highlighting}[]
\FunctionTok{str}\NormalTok{(}\FunctionTok{as.numeric}\NormalTok{(vect))}
\end{Highlighting}
\end{Shaded}

\begin{verbatim}
##  num [1:6] 1 0 1 0 1 1
\end{verbatim}

\begin{Shaded}
\begin{Highlighting}[]
\FunctionTok{cat}\NormalTok{(}\StringTok{"}\SpecialCharTok{\textbackslash{}n}\StringTok{logical to numeric}\SpecialCharTok{\textbackslash{}n}\StringTok{"}\NormalTok{)}
\end{Highlighting}
\end{Shaded}

\begin{verbatim}
## 
## logical to numeric
\end{verbatim}

\begin{Shaded}
\begin{Highlighting}[]
\NormalTok{vect}
\end{Highlighting}
\end{Shaded}

\begin{verbatim}
## [1]  TRUE FALSE  TRUE FALSE  TRUE  TRUE
\end{verbatim}

\begin{Shaded}
\begin{Highlighting}[]
\FunctionTok{as.numeric}\NormalTok{(vect)}
\end{Highlighting}
\end{Shaded}

\begin{verbatim}
## [1] 1 0 1 0 1 1
\end{verbatim}

\begin{Shaded}
\begin{Highlighting}[]
\NormalTok{vect }\OtherTok{\textless{}{-}} \FunctionTok{c}\NormalTok{(}\StringTok{"Dave"}\NormalTok{, }\StringTok{"Vlad"}\NormalTok{,}\StringTok{"Chuck"}\NormalTok{,}\StringTok{"Larry"}\NormalTok{,}\StringTok{"Bob"}\NormalTok{,}\StringTok{"Mary"}\NormalTok{,}\StringTok{"Doug"}\NormalTok{,}\StringTok{"Daryll"}\NormalTok{,}\StringTok{"Ron"}\NormalTok{,}\StringTok{"Sue"}\NormalTok{, }\StringTok{"Laura"}\NormalTok{, }\StringTok{"Rich"}\NormalTok{)}
\FunctionTok{cat}\NormalTok{(}\StringTok{"}\SpecialCharTok{\textbackslash{}n}\StringTok{character to numeric}\SpecialCharTok{\textbackslash{}n}\StringTok{"}\NormalTok{)}
\end{Highlighting}
\end{Shaded}

\begin{verbatim}
## 
## character to numeric
\end{verbatim}

\begin{Shaded}
\begin{Highlighting}[]
\NormalTok{vect}
\end{Highlighting}
\end{Shaded}

\begin{verbatim}
##  [1] "Dave"   "Vlad"   "Chuck"  "Larry"  "Bob"    "Mary"   "Doug"   "Daryll"
##  [9] "Ron"    "Sue"    "Laura"  "Rich"
\end{verbatim}

\begin{Shaded}
\begin{Highlighting}[]
\FunctionTok{as.numeric}\NormalTok{(vect)}
\end{Highlighting}
\end{Shaded}

\begin{verbatim}
## Warning: NAs introduced by coercion
\end{verbatim}

\begin{verbatim}
##  [1] NA NA NA NA NA NA NA NA NA NA NA NA
\end{verbatim}

Oops! Some data can't be coerced!

\hypertarget{as.integer}{%
\subsection{as.integer}\label{as.integer}}

\textbf{as.integer} examples

\begin{Shaded}
\begin{Highlighting}[]
\FunctionTok{cat}\NormalTok{(}\StringTok{"numeric to integer}\SpecialCharTok{\textbackslash{}n}\StringTok{"}\NormalTok{)}
\end{Highlighting}
\end{Shaded}

\begin{verbatim}
## numeric to integer
\end{verbatim}

\begin{Shaded}
\begin{Highlighting}[]
\NormalTok{vect2 }\OtherTok{\textless{}{-}} \FunctionTok{c}\NormalTok{(}\FloatTok{1.1}\NormalTok{,}\SpecialCharTok{{-}}\FloatTok{2.6}\NormalTok{,}\DecValTok{4}\NormalTok{,}\FloatTok{6.6}\NormalTok{,}\FloatTok{2.2}\NormalTok{,}\DecValTok{12}\NormalTok{)}
\NormalTok{vect2}
\end{Highlighting}
\end{Shaded}

\begin{verbatim}
## [1]  1.1 -2.6  4.0  6.6  2.2 12.0
\end{verbatim}

\begin{Shaded}
\begin{Highlighting}[]
\FunctionTok{as.integer}\NormalTok{(vect2)}
\end{Highlighting}
\end{Shaded}

\begin{verbatim}
## [1]  1 -2  4  6  2 12
\end{verbatim}

\emph{as.integer} rounds off numeric values, otherwise has effect like
\textbf{as.numeric}.

\hypertarget{as.logical}{%
\subsection{as.logical}\label{as.logical}}

\textbf{as.logical} examples

\begin{Shaded}
\begin{Highlighting}[]
\FunctionTok{cat}\NormalTok{(}\StringTok{"numeric to logical}\SpecialCharTok{\textbackslash{}n}\StringTok{"}\NormalTok{)}
\end{Highlighting}
\end{Shaded}

\begin{verbatim}
## numeric to logical
\end{verbatim}

\begin{Shaded}
\begin{Highlighting}[]
\NormalTok{vect1 }\OtherTok{\textless{}{-}} \FunctionTok{c}\NormalTok{(}\FloatTok{1.1}\NormalTok{,}\SpecialCharTok{{-}}\FloatTok{2.6}\NormalTok{,}\DecValTok{0}\NormalTok{,}\FloatTok{6.6}\NormalTok{,}\FloatTok{2.2}\NormalTok{,}\DecValTok{12}\NormalTok{)}
\NormalTok{vect1}
\end{Highlighting}
\end{Shaded}

\begin{verbatim}
## [1]  1.1 -2.6  0.0  6.6  2.2 12.0
\end{verbatim}

\begin{Shaded}
\begin{Highlighting}[]
\FunctionTok{as.logical}\NormalTok{(vect1)}
\end{Highlighting}
\end{Shaded}

\begin{verbatim}
## [1]  TRUE  TRUE FALSE  TRUE  TRUE  TRUE
\end{verbatim}

\begin{Shaded}
\begin{Highlighting}[]
\FunctionTok{cat}\NormalTok{(}\StringTok{"}\SpecialCharTok{\textbackslash{}n}\StringTok{integer to logical}\SpecialCharTok{\textbackslash{}n}\StringTok{"}\NormalTok{)}
\end{Highlighting}
\end{Shaded}

\begin{verbatim}
## 
## integer to logical
\end{verbatim}

\begin{Shaded}
\begin{Highlighting}[]
\NormalTok{vect2}
\end{Highlighting}
\end{Shaded}

\begin{verbatim}
## [1]  1.1 -2.6  4.0  6.6  2.2 12.0
\end{verbatim}

\begin{Shaded}
\begin{Highlighting}[]
\FunctionTok{as.logical}\NormalTok{(vect2)}
\end{Highlighting}
\end{Shaded}

\begin{verbatim}
## [1] TRUE TRUE TRUE TRUE TRUE TRUE
\end{verbatim}

\begin{Shaded}
\begin{Highlighting}[]
\FunctionTok{cat}\NormalTok{(}\StringTok{"}\SpecialCharTok{\textbackslash{}n}\StringTok{character to logical}\SpecialCharTok{\textbackslash{}n}\StringTok{"}\NormalTok{)}
\end{Highlighting}
\end{Shaded}

\begin{verbatim}
## 
## character to logical
\end{verbatim}

\begin{Shaded}
\begin{Highlighting}[]
\NormalTok{vect4 }\OtherTok{\textless{}{-}} \FunctionTok{c}\NormalTok{(}\StringTok{"Dave"}\NormalTok{, }\StringTok{"Vlad"}\NormalTok{,}\StringTok{"Chuck"}\NormalTok{,}\StringTok{"Larry"}\NormalTok{,}\StringTok{"Bob"}\NormalTok{,}\StringTok{"Mary"}\NormalTok{,}\StringTok{"Doug"}\NormalTok{,}\StringTok{"Daryll"}\NormalTok{,}\StringTok{"Ron"}\NormalTok{,}\StringTok{"Sue"}\NormalTok{, }\StringTok{"Laura"}\NormalTok{, }\StringTok{"TRUE"}\NormalTok{)}
\NormalTok{vect4}
\end{Highlighting}
\end{Shaded}

\begin{verbatim}
##  [1] "Dave"   "Vlad"   "Chuck"  "Larry"  "Bob"    "Mary"   "Doug"   "Daryll"
##  [9] "Ron"    "Sue"    "Laura"  "TRUE"
\end{verbatim}

\begin{Shaded}
\begin{Highlighting}[]
\FunctionTok{as.logical}\NormalTok{(vect4)}
\end{Highlighting}
\end{Shaded}

\begin{verbatim}
##  [1]   NA   NA   NA   NA   NA   NA   NA   NA   NA   NA   NA TRUE
\end{verbatim}

\hypertarget{coercion-of-mixed-data}{%
\subsection{coercion of mixed data}\label{coercion-of-mixed-data}}

Finally, data are also coerced when creating a vector, if necessary,
since vectors can only contain one type of data.

\begin{Shaded}
\begin{Highlighting}[]
\NormalTok{vect5 }\OtherTok{\textless{}{-}} \FunctionTok{c}\NormalTok{(T, F, }\DecValTok{1}\NormalTok{, T)}
\FunctionTok{str}\NormalTok{(vect5)}
\end{Highlighting}
\end{Shaded}

\begin{verbatim}
##  num [1:4] 1 0 1 1
\end{verbatim}

\begin{Shaded}
\begin{Highlighting}[]
\FunctionTok{cat}\NormalTok{(}\StringTok{"}\SpecialCharTok{\textbackslash{}n}\StringTok{"}\NormalTok{)}
\end{Highlighting}
\end{Shaded}

\begin{Shaded}
\begin{Highlighting}[]
\NormalTok{vect6 }\OtherTok{\textless{}{-}} \FunctionTok{c}\NormalTok{(T, F, }\StringTok{"bob"}\NormalTok{, T)}
\FunctionTok{str}\NormalTok{(vect6)}
\end{Highlighting}
\end{Shaded}

\begin{verbatim}
##  chr [1:4] "TRUE" "FALSE" "bob" "TRUE"
\end{verbatim}

\begin{Shaded}
\begin{Highlighting}[]
\FunctionTok{cat}\NormalTok{(}\StringTok{"}\SpecialCharTok{\textbackslash{}n}\StringTok{"}\NormalTok{)}
\end{Highlighting}
\end{Shaded}

\begin{Shaded}
\begin{Highlighting}[]
\NormalTok{vect7 }\OtherTok{\textless{}{-}} \FunctionTok{c}\NormalTok{(}\DecValTok{1}\NormalTok{, }\FloatTok{2.3}\NormalTok{, }\StringTok{"bob"}\NormalTok{,}\DecValTok{7}\NormalTok{)}
\FunctionTok{str}\NormalTok{(vect7)}
\end{Highlighting}
\end{Shaded}

\begin{verbatim}
##  chr [1:4] "1" "2.3" "bob" "7"
\end{verbatim}

Logical is coerced to numeric. Everything is coerced to character if
there is a character element. As we have seen previously, when you
combine numeric and integers, a numeric vector results.

\textbf{Notes}

\begin{itemize}
\tightlist
\item
  Where coersion fails, NA values result.
\item
  MANY functions coerse data and objects into what is needed. This can
  be both useful and dangerous!
\end{itemize}

\hypertarget{section-26}{%
\section{--------------}\label{section-26}}

\hypertarget{matrices}{%
\section{Matrices}\label{matrices}}

\hypertarget{section-27}{%
\section{--------------}\label{section-27}}

A \textbf{matrix} in R is a 2-dimensional data structure. A
\textbf{matrix} holds a n by m grid of values, with each of the two
dimensions (practically) any length. As with atomic vectors, a matrix
must contain values of just one type.

\hypertarget{section-28}{%
\section{--------------}\label{section-28}}

\hypertarget{creating-a-matrix}{%
\subsection{Creating a matrix}\label{creating-a-matrix}}

\hypertarget{section-29}{%
\section{--------------}\label{section-29}}

You can create a \textbf{matrix} using the \textbf{matrix} function.

\begin{Shaded}
\begin{Highlighting}[]
\CommentTok{\# \textquotesingle{}data\textquotesingle{} is a vector}
\CommentTok{\# \textquotesingle{}nrow\textquotesingle{} gives the number of rows}
\CommentTok{\# \textquotesingle{}ncol\textquotesingle{} gives the number of columns}
\CommentTok{\# \textquotesingle{}byrow\textquotesingle{} by default is FALSE, so data is read in column{-}wise}
\NormalTok{vect }\OtherTok{\textless{}{-}} \FunctionTok{c}\NormalTok{(}\DecValTok{1}\NormalTok{,}\SpecialCharTok{{-}}\DecValTok{2}\NormalTok{,}\DecValTok{4}\NormalTok{,}\DecValTok{6}\NormalTok{,}\FloatTok{2.2}\NormalTok{,}\DecValTok{12}\NormalTok{)}
\NormalTok{mat }\OtherTok{\textless{}{-}} \FunctionTok{matrix}\NormalTok{(}\AttributeTok{data =}\NormalTok{ vect, }\AttributeTok{nrow =} \DecValTok{3}\NormalTok{, }\AttributeTok{ncol =} \DecValTok{2}\NormalTok{)}
\NormalTok{mat}
\end{Highlighting}
\end{Shaded}

\begin{verbatim}
##      [,1] [,2]
## [1,]    1  6.0
## [2,]   -2  2.2
## [3,]    4 12.0
\end{verbatim}

\hypertarget{examine-matrix-properties}{%
\subsection{Examine matrix properties}\label{examine-matrix-properties}}

We can examine the properties of this matrix data object.

\begin{Shaded}
\begin{Highlighting}[]
\CommentTok{\# dim function gives the dimensions of a matrix}
\FunctionTok{cat}\NormalTok{(}\StringTok{"matrix row{-}column dimensions:"}\NormalTok{, }\FunctionTok{dim}\NormalTok{(mat),}\StringTok{"}\SpecialCharTok{\textbackslash{}n}\StringTok{"}\NormalTok{)}
\end{Highlighting}
\end{Shaded}

\begin{verbatim}
## matrix row-column dimensions: 3 2
\end{verbatim}

\begin{Shaded}
\begin{Highlighting}[]
\CommentTok{\# nrow gives number of rows}
\FunctionTok{cat}\NormalTok{(}\StringTok{"}\SpecialCharTok{\textbackslash{}n}\StringTok{matrix row dimension:"}\NormalTok{, }\FunctionTok{nrow}\NormalTok{(mat),}\StringTok{"}\SpecialCharTok{\textbackslash{}n}\StringTok{"}\NormalTok{)}
\end{Highlighting}
\end{Shaded}

\begin{verbatim}
## 
## matrix row dimension: 3
\end{verbatim}

\begin{Shaded}
\begin{Highlighting}[]
\CommentTok{\# ncol gives number of rows}
\FunctionTok{cat}\NormalTok{(}\StringTok{"}\SpecialCharTok{\textbackslash{}n}\StringTok{matrix column dimension:"}\NormalTok{, }\FunctionTok{ncol}\NormalTok{(mat),}\StringTok{"}\SpecialCharTok{\textbackslash{}n}\StringTok{"}\NormalTok{)}
\end{Highlighting}
\end{Shaded}

\begin{verbatim}
## 
## matrix column dimension: 2
\end{verbatim}

\begin{Shaded}
\begin{Highlighting}[]
\CommentTok{\# typeof gives use the type of data}
\FunctionTok{cat}\NormalTok{(}\StringTok{"}\SpecialCharTok{\textbackslash{}n}\StringTok{matrix data type:"}\NormalTok{, }\FunctionTok{typeof}\NormalTok{(mat),}\StringTok{"}\SpecialCharTok{\textbackslash{}n}\StringTok{"}\NormalTok{)}
\end{Highlighting}
\end{Shaded}

\begin{verbatim}
## 
## matrix data type: double
\end{verbatim}

\begin{Shaded}
\begin{Highlighting}[]
\CommentTok{\# str is also useful}
\FunctionTok{cat}\NormalTok{(}\StringTok{"}\SpecialCharTok{\textbackslash{}n}\StringTok{matrix structure:}\SpecialCharTok{\textbackslash{}n}\StringTok{"}\NormalTok{)}
\end{Highlighting}
\end{Shaded}

\begin{verbatim}
## 
## matrix structure:
\end{verbatim}

\begin{Shaded}
\begin{Highlighting}[]
\FunctionTok{str}\NormalTok{(mat)}
\end{Highlighting}
\end{Shaded}

\begin{verbatim}
##  num [1:3, 1:2] 1 -2 4 6 2.2 12
\end{verbatim}

Note that you use \emph{dim} for matrices, \emph{length} to extract the
dimensions of vectors.

\hypertarget{name-attributes-for-matrix}{%
\subsection{name attributes for
matrix}\label{name-attributes-for-matrix}}

We can also name rows or columns by assignment.

\begin{Shaded}
\begin{Highlighting}[]
\CommentTok{\#first look at attributes}
\FunctionTok{attributes}\NormalTok{(mat)}
\end{Highlighting}
\end{Shaded}

\begin{verbatim}
## $dim
## [1] 3 2
\end{verbatim}

\begin{Shaded}
\begin{Highlighting}[]
\FunctionTok{cat}\NormalTok{(}\StringTok{"}\SpecialCharTok{\textbackslash{}n}\StringTok{"}\NormalTok{)    }\CommentTok{\# blank line}
\end{Highlighting}
\end{Shaded}

\begin{Shaded}
\begin{Highlighting}[]
\CommentTok{\#colnames and rownames functions are used to address column and row name attributes}
\FunctionTok{colnames}\NormalTok{(mat) }\OtherTok{\textless{}{-}} \FunctionTok{c}\NormalTok{(}\StringTok{"Var1"}\NormalTok{,}\StringTok{"Var2"}\NormalTok{)}
\NormalTok{mat}
\end{Highlighting}
\end{Shaded}

\begin{verbatim}
##      Var1 Var2
## [1,]    1  6.0
## [2,]   -2  2.2
## [3,]    4 12.0
\end{verbatim}

\begin{Shaded}
\begin{Highlighting}[]
\FunctionTok{cat}\NormalTok{(}\StringTok{"}\SpecialCharTok{\textbackslash{}n}\StringTok{"}\NormalTok{)}
\end{Highlighting}
\end{Shaded}

\begin{Shaded}
\begin{Highlighting}[]
\FunctionTok{rownames}\NormalTok{(mat) }\OtherTok{\textless{}{-}} \FunctionTok{c}\NormalTok{(}\StringTok{"a"}\NormalTok{,}\StringTok{"b"}\NormalTok{,}\StringTok{"c"}\NormalTok{)}
\NormalTok{mat}
\end{Highlighting}
\end{Shaded}

\begin{verbatim}
##   Var1 Var2
## a    1  6.0
## b   -2  2.2
## c    4 12.0
\end{verbatim}

\begin{Shaded}
\begin{Highlighting}[]
\CommentTok{\#now look at attributes again}
\FunctionTok{cat}\NormalTok{(}\StringTok{"}\SpecialCharTok{\textbackslash{}n}\StringTok{"}\NormalTok{)    }\CommentTok{\# blank line}
\end{Highlighting}
\end{Shaded}

\begin{Shaded}
\begin{Highlighting}[]
\FunctionTok{attributes}\NormalTok{(mat)}
\end{Highlighting}
\end{Shaded}

\begin{verbatim}
## $dim
## [1] 3 2
## 
## $dimnames
## $dimnames[[1]]
## [1] "a" "b" "c"
## 
## $dimnames[[2]]
## [1] "Var1" "Var2"
\end{verbatim}

\hypertarget{use-seq-to-generate-matrix-data}{%
\section{\texorpdfstring{use \textbf{seq} to generate matrix
data}{use seq to generate matrix data}}\label{use-seq-to-generate-matrix-data}}

Now we can create a larger matrix with sequenced numbers.

\begin{Shaded}
\begin{Highlighting}[]
\CommentTok{\# create a larger matrix using sequenced numbers}
\NormalTok{mat }\OtherTok{\textless{}{-}} \FunctionTok{matrix}\NormalTok{(}\AttributeTok{data =} \DecValTok{1}\SpecialCharTok{:}\DecValTok{20}\NormalTok{, }\AttributeTok{nrow =} \DecValTok{4}\NormalTok{, }\AttributeTok{ncol =} \DecValTok{5}\NormalTok{)}
\NormalTok{mat}
\end{Highlighting}
\end{Shaded}

\begin{verbatim}
##      [,1] [,2] [,3] [,4] [,5]
## [1,]    1    5    9   13   17
## [2,]    2    6   10   14   18
## [3,]    3    7   11   15   19
## [4,]    4    8   12   16   20
\end{verbatim}

\begin{Shaded}
\begin{Highlighting}[]
\FunctionTok{cat}\NormalTok{(}\StringTok{"}\SpecialCharTok{\textbackslash{}n}\StringTok{"}\NormalTok{)}
\end{Highlighting}
\end{Shaded}

\begin{Shaded}
\begin{Highlighting}[]
\FunctionTok{str}\NormalTok{(mat)}
\end{Highlighting}
\end{Shaded}

\begin{verbatim}
##  int [1:4, 1:5] 1 2 3 4 5 6 7 8 9 10 ...
\end{verbatim}

\hypertarget{read-by-rows}{%
\subsection{Read by rows}\label{read-by-rows}}

What happens if we read in by rows?

\begin{Shaded}
\begin{Highlighting}[]
\CommentTok{\# create a larger matrix using sequenced numbers, read in by row}
\NormalTok{mat }\OtherTok{\textless{}{-}} \FunctionTok{matrix}\NormalTok{(}\AttributeTok{data =} \DecValTok{1}\SpecialCharTok{:}\DecValTok{20}\NormalTok{, }\AttributeTok{nrow =} \DecValTok{4}\NormalTok{, }\AttributeTok{ncol =} \DecValTok{5}\NormalTok{, }\AttributeTok{byrow=}\NormalTok{T)}
\NormalTok{mat}
\end{Highlighting}
\end{Shaded}

\begin{verbatim}
##      [,1] [,2] [,3] [,4] [,5]
## [1,]    1    2    3    4    5
## [2,]    6    7    8    9   10
## [3,]   11   12   13   14   15
## [4,]   16   17   18   19   20
\end{verbatim}

\begin{Shaded}
\begin{Highlighting}[]
\FunctionTok{cat}\NormalTok{(}\StringTok{"}\SpecialCharTok{\textbackslash{}n}\StringTok{"}\NormalTok{)}
\end{Highlighting}
\end{Shaded}

\begin{Shaded}
\begin{Highlighting}[]
\FunctionTok{str}\NormalTok{(mat)}
\end{Highlighting}
\end{Shaded}

\begin{verbatim}
##  int [1:4, 1:5] 1 6 11 16 2 7 12 17 3 8 ...
\end{verbatim}

Note that matrix data is always stored column-wise, even if read in
row-wise.

\hypertarget{a-few-matrix-operations}{%
\subsection{A few matrix operations}\label{a-few-matrix-operations}}

There are many useful matrix operations.

\begin{Shaded}
\begin{Highlighting}[]
\NormalTok{mat}
\end{Highlighting}
\end{Shaded}

\begin{verbatim}
##      [,1] [,2] [,3] [,4] [,5]
## [1,]    1    2    3    4    5
## [2,]    6    7    8    9   10
## [3,]   11   12   13   14   15
## [4,]   16   17   18   19   20
\end{verbatim}

\begin{Shaded}
\begin{Highlighting}[]
\FunctionTok{cat}\NormalTok{(}\StringTok{"}\SpecialCharTok{\textbackslash{}n}\StringTok{Transpose matrix}\SpecialCharTok{\textbackslash{}n}\StringTok{"}\NormalTok{)}
\end{Highlighting}
\end{Shaded}

\begin{verbatim}
## 
## Transpose matrix
\end{verbatim}

\begin{Shaded}
\begin{Highlighting}[]
\FunctionTok{t}\NormalTok{(mat)    }\CommentTok{\# transpose matrix}
\end{Highlighting}
\end{Shaded}

\begin{verbatim}
##      [,1] [,2] [,3] [,4]
## [1,]    1    6   11   16
## [2,]    2    7   12   17
## [3,]    3    8   13   18
## [4,]    4    9   14   19
## [5,]    5   10   15   20
\end{verbatim}

\begin{Shaded}
\begin{Highlighting}[]
\CommentTok{\# num [1:3, 1:2] 1 {-}2 4 6 2.2 12}

\FunctionTok{cat}\NormalTok{(}\StringTok{"}\SpecialCharTok{\textbackslash{}n}\StringTok{Another way to reshape matrix}\SpecialCharTok{\textbackslash{}n}\StringTok{"}\NormalTok{)}
\end{Highlighting}
\end{Shaded}

\begin{verbatim}
## 
## Another way to reshape matrix
\end{verbatim}

\begin{Shaded}
\begin{Highlighting}[]
\FunctionTok{dim}\NormalTok{(mat) }\OtherTok{\textless{}{-}} \FunctionTok{c}\NormalTok{(}\DecValTok{5}\NormalTok{,}\DecValTok{4}\NormalTok{)}
\NormalTok{mat}
\end{Highlighting}
\end{Shaded}

\begin{verbatim}
##      [,1] [,2] [,3] [,4]
## [1,]    1    7   13   19
## [2,]    6   12   18    5
## [3,]   11   17    4   10
## [4,]   16    3    9   15
## [5,]    2    8   14   20
\end{verbatim}

\begin{Shaded}
\begin{Highlighting}[]
\CommentTok{\#cat("\textbackslash{}nAnother way to reshape matrix\textbackslash{}n")}
\CommentTok{\#dim(mat1) \textless{}{-} c(4,3)}
\CommentTok{\#mat1}

\FunctionTok{cat}\NormalTok{(}\StringTok{"}\SpecialCharTok{\textbackslash{}n}\StringTok{Create square matrix}\SpecialCharTok{\textbackslash{}n}\StringTok{"}\NormalTok{)}
\end{Highlighting}
\end{Shaded}

\begin{verbatim}
## 
## Create square matrix
\end{verbatim}

\begin{Shaded}
\begin{Highlighting}[]
\NormalTok{mat }\OtherTok{\textless{}{-}} \FunctionTok{matrix}\NormalTok{(}\FunctionTok{c}\NormalTok{(}\DecValTok{1}\NormalTok{,}\DecValTok{3}\NormalTok{,}\DecValTok{5}\NormalTok{,}\DecValTok{2}\NormalTok{,}\DecValTok{0}\NormalTok{,}\DecValTok{0}\NormalTok{,}\SpecialCharTok{{-}}\DecValTok{1}\NormalTok{,}\DecValTok{6}\NormalTok{,}\FloatTok{0.5}\NormalTok{),}\AttributeTok{nrow=}\DecValTok{3}\NormalTok{,}\AttributeTok{ncol=}\DecValTok{3}\NormalTok{)}
\NormalTok{mat}
\end{Highlighting}
\end{Shaded}

\begin{verbatim}
##      [,1] [,2] [,3]
## [1,]    1    2 -1.0
## [2,]    3    0  6.0
## [3,]    5    0  0.5
\end{verbatim}

\begin{Shaded}
\begin{Highlighting}[]
\FunctionTok{cat}\NormalTok{(}\StringTok{"}\SpecialCharTok{\textbackslash{}n}\StringTok{Invert square matrix}\SpecialCharTok{\textbackslash{}n}\StringTok{"}\NormalTok{)}
\end{Highlighting}
\end{Shaded}

\begin{verbatim}
## 
## Invert square matrix
\end{verbatim}

\begin{Shaded}
\begin{Highlighting}[]
\FunctionTok{solve}\NormalTok{(mat) }
\end{Highlighting}
\end{Shaded}

\begin{verbatim}
##      [,1]        [,2]       [,3]
## [1,]  0.0 -0.01754386  0.2105263
## [2,]  0.5  0.09649123 -0.1578947
## [3,]  0.0  0.17543860 -0.1052632
\end{verbatim}

\begin{Shaded}
\begin{Highlighting}[]
\FunctionTok{cat}\NormalTok{(}\StringTok{"}\SpecialCharTok{\textbackslash{}n}\StringTok{multiply matrices}\SpecialCharTok{\textbackslash{}n}\StringTok{"}\NormalTok{)}
\end{Highlighting}
\end{Shaded}

\begin{verbatim}
## 
## multiply matrices
\end{verbatim}

\begin{Shaded}
\begin{Highlighting}[]
\FunctionTok{round}\NormalTok{(}\FunctionTok{solve}\NormalTok{(mat) }\SpecialCharTok{\%*\%}\NormalTok{ mat, }\DecValTok{3}\NormalTok{)  }\CommentTok{\# round off multiplication}
\end{Highlighting}
\end{Shaded}

\begin{verbatim}
##      [,1] [,2] [,3]
## [1,]    1    0    0
## [2,]    0    1    0
## [3,]    0    0    1
\end{verbatim}

\hypertarget{functions}{%
\subsection{functions}\label{functions}}

For expediency, you can write your own functions

\begin{Shaded}
\begin{Highlighting}[]
\NormalTok{?sq   }\CommentTok{\# there is no sq function, but we can create one}
\end{Highlighting}
\end{Shaded}

\begin{verbatim}
## No documentation for 'sq' in specified packages and libraries:
## you could try '??sq'
\end{verbatim}

\begin{Shaded}
\begin{Highlighting}[]
\NormalTok{sq }\OtherTok{\textless{}{-}} \ControlFlowTok{function}\NormalTok{(x) \{}
\NormalTok{  y }\OtherTok{\textless{}{-}}\NormalTok{ x}\SpecialCharTok{*}\NormalTok{x}
  \FunctionTok{return}\NormalTok{(y)}
\NormalTok{\}}

\NormalTok{x }\OtherTok{\textless{}{-}} \DecValTok{4}
\FunctionTok{sq}\NormalTok{(}\DecValTok{4}\NormalTok{)}
\end{Highlighting}
\end{Shaded}

\begin{verbatim}
## [1] 16
\end{verbatim}

Notice that there is no y or x value in our environment. Variables
created within a function are not saved outside of the function. You
need to return these values or otherwise save them if needed. There is a
new sq object in our environment.

\begin{Shaded}
\begin{Highlighting}[]
\CommentTok{\#source("Rlib.txt")}
\end{Highlighting}
\end{Shaded}


\end{document}
